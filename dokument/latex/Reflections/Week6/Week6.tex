\documentclass{scrartcl}

\setlength{\parskip}{\baselineskip}
\setlength{\parindent}{0em}

\usepackage{polyglossia,graphicx}
\setdefaultlanguage{swedish}
\usepackage{csquotes}

\begin{document}

\title{Gruppreflektion för vecka 6}
\author{Group Joy}
\date{8 oktober 2021}
\maketitle

\section{Customer Value and Scope}
The application being developed is gradually forming into something that is more or less defined as the project goal in the beginning of the course.

Prioritizing some of the project's design decisions made flexibility in project development quite low and is something that we could improve. Some steps that could lead the project to grow more organically is to react promptly to the developer’s needs - are the user stories really formed for those who are going to work on them? Are they allowed to be reformatted, erased and has the week’s scrum master made the effort to see the user stories from a wider perspective - in which way are they contributing to creating value. This is something we could work on.

\section{Design decisions and product structure}
D3 is a highly versatile library that so far works great for our purposes. Although a little complicated to learn, which may have slowed our process, it gives us power to make the website look and behave just as we imagined early in the course. It also gives us lots of room to add more features in the future of any kind related to our topic. 
For a couple of weeks we have had one additional short meeting, reflecting briefly on our work and discussing our solutions to make sure that these will fit well together. This is an opportunity to get some quick help with the techniques we apply (and are not necessarily a custom to yet) and to make sure that someone struggling with a user story is helped before the end of the sprint.

\section{Application of Scrum}
The project is very “horisontal” by its nature, it is very difficult to work on different parts of the project because they are all dependent on each other. This leads to the need for longer sprints and work in larger groups or radically reorganising the project, for work in smaller groups and shorter sprints which unfortunately cannot be done because the decisions that were made in the beginning are difficult to change.

Switching the role of product owner and scrum master within the group works well as to get the idea how those roles influence the project, but it is still leading to problems when depending on decisions that were made by other team members in the past, how much is OK to change in each run? Until now we were very understandable in continuing the work of recent POs/SMs, but it could be a good idea to use the freedom these positions offer for the benefit of reaching the epics more efficiently.

The concept of agile development and scrum is, as far as the results of the common work on our project show, not as easy to learn as it might seem. The social dynamics within the group, different forms of hierarchies, if it is possible to call them as such, levels of understanding the technical problems of the project, ability and willingness to learn, play a significant role in how the project was driven until now. Is the “nature” of the way each of us think and deal with problems more structured as a waterfall than a dynamic system that structures itself as the surrounding changes, more like in case of scrum? To learn scrum is therefore possibly much more about changing the way of seeing a problem, than being focused on reaching the goal. A shift in that direction would include constant reflection about and during the working process, within the group, furthermore within ourselves. We will use the experience of meeting the colleagues in a formal and informal context to build the reflecting spirit which we will try to strengthen until the end of the course.


\end{document}