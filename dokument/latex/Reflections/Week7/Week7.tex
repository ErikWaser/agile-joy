\documentclass{scrartcl}

\setlength{\parskip}{\baselineskip}
\setlength{\parindent}{0em}

\usepackage{polyglossia,graphicx}
\setdefaultlanguage{english}
\usepackage{csquotes}

\begin{document}

\title{Gruppreflektion för vecka 7}
\author{Group Joy}
\date{8 oktober 2021}
\maketitle

\section{Customer Value and Scope}
As the project is approaching the final stretch and our project is nearing completion it seems that the user stories are all pretty small and we should be able to increase the number of stories we could get done because of it. We have however ran into a different problem. Since our project is fairly tightly restricted we find ourselves in a situation of some stories being built on each other. As we commented last week this is most likely because we dimensioned the project wrong. The project is probably not large enough for the number of people working on it. Either that or we need to work on how we build our user stories. The result is that while each user story might generate value, they are dificult to split up among different people. 

One solution that we have discussed might be to try and expand the scope of the project, for instance displaying diferent data than just covid numbers. Perhaps we could use our program to view the number of people who owned a car for instance. Or maybe we could use it to show the percentages of female polititians, or avarege income. It has a lot of potential as a visual aid for a lot of differend data as long as we can find it. 


\section{Social Contract and Effort}


We are still satisfied with the social contract however it is still not fully followed, for example the meeting participation still hasn't improved. Lately we have decided to book meetings a day before having them, this made it problematic because not everyone is checking the communication channel on a regular basis and this have made some people miss the meetings entirely due to not knowing we were having one but also didn't give them any time to inform the others if they couldn't make it.

The ideal situation would be that we decide when to have the next meeting during the current meeting or a few days before giving everyone enough time to check discord. Expect everyone to inform the team if they can't make it to the meeting and follow the social contract better.

In order to avoid this in the future we should make sure that everyone tries to check discord more regularly and to make sure to write in the channel when the next meeting is and make sure to notify everyone but also to expect people to inform the rest of the team if they cant make it.



%Have meeting participation bettered?
%Just nu saknas ju två...
%Har varit ganska dåligt? sämre på att hålla koll på möten

%Is our comunication good? 
%Det är viktigt att vi anonserar möten 
%och att folk söker upp vad som sades

%Do we feel like someone is not contributing enough? if so, what do we do? 


\section{Design decisions and product structure}
The project is taking form nicely. The D3 library is working great for our purposes and the visual design is sleek and looks nice. During the week we restructured the repo. The javascript is devided into neat seperate files with a markdown in the folder explaining what each file does. We have also made sure that we can use npm (Node Package Manager) to quickly launch our code.

%SOMETHING ABOUT CODE STANDARD @SLAGET WHERE WE AT MED DET?

% SOMETHING ABOUT COMMENTING 

%Visual design of the project? How is it structured? Vad har vi för kodstandard? filepath structure? 

\section{Application of Scrum}
Over the week we have talked abour things that we wish we would have done before anything else. One of those things is that it might have been a good idea to have the SM and PO be non-rotating. As we discussed last week we have had some issues in what they are supposed to do and not do. In having each one of us take a week each to try and make sure everyone got to try it, we may have simply made sure that none of us got to properly try it. This is something we plan on discussing in the meeting to start the final sprint and see if we may want to keep the schedule going or if we want to keep the current ones for next week as well. It may be a pointless change to employ so late in the course but we could always try. 

We have also talked about our KPIs. We decided on our final KPIs very late (read more about them in the file named KPI.md in the github repo). We decided on shirt sizes fairly quickly and comparing estimated and actual velocity soon afrter that. But our final KPI, analysing User Points for how much value we actualy added to the users themselves as opposed to the developers. If we had all of our KPIs ready and properly defined to start off then it would have been easier to employ them and have a more agile workflow. 

We have in short pondered wether we should have made more time and put down more effort on the agile part of our project. We have in general focused a lot on the project itself and probably less on the agile process than we should have. 


\end{document}