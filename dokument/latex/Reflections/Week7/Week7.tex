\documentclass{scrartcl}

\setlength{\parskip}{\baselineskip}
\setlength{\parindent}{0em}

\usepackage{polyglossia,graphicx}
\setdefaultlanguage{english}
\usepackage{csquotes}

\begin{document}

\title{Gruppreflektion för vecka 7}
\author{Group Joy}
\date{8 oktober 2021}
\maketitle

\section{Customer Value and Scope}
As the project is approaching the final stretch and our project is nearing completion it seems that the user stories are all pretty small and we should be able to increase the number of stories we could get done because of it. We have however ran into a different problem. Since our project is fairly tightly restricted we find ourselves in a situation of some stories being built on each other. As we commented last week this is most likely because we dimensioned the project wrong. The project is probably not large enough for the number of people working on it. Either that or we need to work on how we build our user stories. The result is that while each user story might generate value, they are dificult to split up among different people. 

One solution that we have discussed might be to try and expand the scope of the project, for instance displaying diferent data than just covid numbers. Perhaps we could use our program to view the number of people who owned a car for instance. Or maybe we could use it to show the percentages of female polititians, or avarege income. It has a lot of potential as a visual aid for a lot of differend data as long as we can find it. 


\section{Social Contract and Effort}
Have meeting participation bettered? 
Is our comunication good? 
Do we feel like someone is not contributing enough? if so, what do we do?

\section{Design decisions and product structure}
Visual design of the project? 
How is it structured? 

\section{Application of Scrum}
Do we have the meetings? 
do we folow the scrum principles? 
Should we perhaps focus/ have focused on the scrum principles more than the code "meat" of the project?


\end{document}